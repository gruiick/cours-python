\documentclass{beamer}

\usepackage[utf8]{inputenc}
\usepackage{hyperref}
\usepackage{minted}

\usetheme{Madrid}

\hypersetup{colorlinks=true}

\title{Introduction à la programmation avec Python (chapitre 2)}
\author{Dimitri Merejkowsky}
\institute{E2L}

\begin{document}

\frame{\titlepage}

\begin{frame}

\frametitle{Session 2}

\end{frame}

\begin{frame}

Note: \\~\\


Les sources sont
\href{https://github.com/E2L/cours-python/tree/master/sources}{sur GitHub}. \\~\\


Mais il vaut mieux recopier le code vous-mêmes.

\end{frame}


\begin{frame}

\frametitle{Plan}

\begin{itemize}
  \item Retour sur le chapitre 1
  \item Structures de données
  \item Fonctions
\end{itemize}

\end{frame}

\begin{frame}[fragile]
    \frametitle{Retour sur input()}



\begin{minted}{python}
# À adapter
import random
secret = random.randint()

print("Devine le nombre auquel je pense")
while True:
    reponse = input("Ta réponse: ")
    response = int(response)
    ...
\end{minted}

\end{frame}



\begin{frame}[fragile]
    \frametitle{Retour sur print}

\begin{minted}{python}
>>> a = 1
>>> b = 2
>>> print("a is", a, "b is", b)
a is 1, b is 2
\end{minted}

\begin{itemize}
  \item On peut fournir plusieurs valeurs, séparées par des virgules
  \item \textit{print()} insère des espaces
  \item et va à la ligne
\end{itemize}

\end{frame}

\begin{frame}[fragile]
  \frametitle{Retour sur print (2)}

\begin{minted}{python}
a = 1
b = 2
print("a=", 1, "b=2", sep="", end="")
\end{minted}

\end{frame}

\begin{frame}[fragile]
  \frametitle{Retour sur les strings}
\end{frame}

\begin{frame}[fragile]
  \frametitle{Répéter une string}

\begin{minted}{python}
>>> "argh " * 3
argh argh argh
\end{minted}

\end{frame}


\begin{frame}[fragile]
  \frametitle{Faire une longue string sur plusieurs lignes}

\begin{minted}{python}
poeme = """
Ceci est un poème

Qui contient "des quotes"
Et parle d'autre choses ...
"""
\end{minted}

\begin{block}{Note}
Marche aussi avec des "triples-simple-quotes", mais c'est moins lisible :P
\end{block}
\end{frame}






%---

%# Concaténer des strings

     %!python
      %message = (
         %"Première ligne\n"
         %"Deuxième ligne\n"
      %)

%Les parenthèse permettent d'aller à la ligne dans le code :)

%---

%# Slicer des strings

    %!pycon
      %>>> message = "Bonjour, monde !"
      %>>> message[0]  # ça commence à zéro
      %"B"
      %>>> message[15]
      %"!"
      %>>>> message[-1]  # compter à l'envers
      %"!"

%---

%# Slicer des strings (2)

     %!pycon
      %>>> message = "Bonjour, monde !"
      %>>> message[1:4]  # début, fin
      %'onj'
      %>>> message[:7] # début implicite
      %'Bonjour'
      %>>> message[9:-2] # fin négative
      %'monde'

%---

%# Listes

%---

%# Créer une liste

     %!pycon
      %>>> my_list = []  # liste vide
      %>>> primes = [2, 3, 5, 7, 11]  # liste d'entiers

%---

%# Listes hétérogènes

%On peut mettre des types différents dans une même liste:

     %!pycon
      %>>> pommes_et_carottes = [True, 2, "three"]

%Et même des listes dans des listes:

     %!pycon
      %>>> liste_de_liste = [[1, 2, 3], ["one", "two", "three"]]

%----

%# Slicer des listes

%Même principe que pour les strings!

     %!pycon
      %>>> liste = [1, 2, 3]
      %>>> liste[0:2]
      %[1, 2]

%---

%# Modifier une liste

     %!pycon
      %>>> liste = [1, 2, 3]
      %>>> liste[1] = 4
      %>>> liste
      %[1, 4, 3]


%*Attention*: ça ne marche pas avec les strings:

     %!pycon
      %>>> message = "Bonjour, monde !"
      %>>> message[-1] = "?"
      %TypeError: 'str' object does not support item assignment

%---

%# Boucles for

%Itérer sur les éléments d'une liste:


     %!python
      %names = ["Alice", "Bob", "Charlie"]
      %for name in names:
         %print("Bonjour", name)

      %Bonjour Alice
      %Bonjour Bob
      %Bonjour Charlie

%---

\end{document}
